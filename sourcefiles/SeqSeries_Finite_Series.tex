**Tutorial Question: Finite Series**

**1. Introduction:**

*   A finite series is the sum of a finite number of terms in a sequence. 
*   In this exercise, we will explore some common types of finite series and their properties.

**2. Exercises:**

**Exercise 1: Arithmetic Series**

*   **a) Define an arithmetic series.** 
    *   *An arithmetic series is the sum of the terms in an arithmetic sequence, where each term differs from the previous term by a constant common difference (d).*
*   **b) Given the arithmetic series: 2 + 5 + 8 + 11 + ... + 29**
      *   Find the common difference (d).
      *   Determine the number of terms (n) in the series.
      *   Calculate the sum of the series using the formula for the sum of an arithmetic series. 

**Exercise 2: Geometric Series**

*   **a) Define a geometric series.**
    *   *A geometric series is the sum of the terms in a geometric sequence, where each term is found by multiplying the previous term by a constant common ratio (r).*
*   **b) Given the geometric series: 3 + 6 + 12 + 24 + ... + 192**
      *   Find the common ratio (r).
      *   Determine the number of terms (n) in the series.
      *   Calculate the sum of the series using the formula for the sum of a geometric series.

**Exercise 3: Application: Sum of Natural Numbers**

*   **a) Find the sum of the first 100 natural numbers (1 + 2 + 3 + ... + 100).**
    *   *Hint: Use the formula for the sum of an arithmetic series.*

**Exercise 4: Application: Binary Representation**

*   **a) Explain how binary numbers can be represented as a sum of powers of 2.**
    *   *Hint: Consider the place values in a binary number.*

**Exercise 5: (Challenge) Nested Sum**

*   **a) Evaluate the following nested sum:** 
      *   ∑<sub>i=1</sub><sup>3</sup> ∑<sub>j=1</sub><sup>2</sup> (i + j)

**3. Hints and Guidance**

*   **Review:** Ensure you understand the definitions of arithmetic and geometric sequences.
*   **Formulas:** 
      *   Sum of an arithmetic series: S<sub>n</sub> = (n/2)(2a + (n-1)d) 
          *   where S<sub>n</sub> is the sum of the first n terms, a is the first term, and d is the common difference.
      *   Sum of a geometric series: S<sub>n</sub> = a(1 - r<sup>n</sup>) / (1 - r) 
          *   where S<sub>n</sub> is the sum of the first n terms, a is the first term, and r is the common ratio.
*   **Programming:** Consider implementing functions in a programming language (like Python) to calculate the sum of arithmetic and geometric series.

**4. Solutions**

*   **Exercise 1:** 
      *   b) d = 3, n = 10, Sum = 155
*   **Exercise 2:** 
      *   b) r = 2, n = 7, Sum = 381
*   **Exercise 3:** 
      *   a) Sum = 5050
*   **Exercise 4:** 
      *   a) Binary numbers can be represented as a sum of powers of 2 based on the positions of the 1's in the binary representation.

This tutorial sheet covers key concepts of finite series, including arithmetic and geometric series, and provides exercises to help students solidify their understanding and apply these concepts. 

**Note:** This is a basic example. You can adjust the difficulty and scope of the exercises based on the specific learning objectives and the level of your students.
