**For undergraduate computer science students studying graph theory, here are some excellent Python packages for creating suitable graphics:**

**1. NetworkX**

*   **Strengths:**
    *   Specifically designed for graph analysis and visualization.
    *   Extensive functionality for creating, manipulating, and analyzing graphs.
    *   Supports various graph types (directed, undirected, weighted).
    *   Provides built-in drawing functions using Matplotlib.
    *   Well-documented and widely used in the scientific community.

*   **Example (Basic Graph):**

```python
import networkx as nx
import matplotlib.pyplot as plt

# Create a graph
G = nx.Graph()
G.add_edges_from([(1, 2), (1, 3), (2, 3), (2, 4), (3, 4)])

# Draw the graph
nx.draw(G, with_labels=True)
plt.show()
```

**2. Matplotlib**

*   **Strengths:**
    *   A general-purpose plotting library with powerful customization options.
    *   Can be used to create various types of plots, including network graphs.
    *   Provides fine-grained control over plot appearance.

*   **Example (Simple Graph):**

```python
import matplotlib.pyplot as plt

# Define node positions
pos = {1: (0, 1), 2: (1, 1), 3: (1, 0), 4: (2, 0)}

# Create a list of edges
edges = [(1, 2), (1, 3), (2, 3), (2, 4), (3, 4)]

# Plot nodes
plt.scatter(x=[pos[node][0] for node in pos], 
            y=[pos[node][1] for node in pos])

# Plot edges
for edge in edges:
    plt.plot([pos[edge[0]][0], pos[edge[1]][0]], 
             [pos[edge[0]][1], pos[edge[1]][1]], 
             color='black')

plt.show()
```

**3. PyVis**

*   **Strengths:**
    *   Creates interactive and visually appealing network graphs.
    *   Easy to use and generate HTML output for web-based visualizations.
    *   Offers features like node and edge customization, force-directed layouts, and zooming.

*   **Example (Interactive Graph):**

```python
from pyvis.network import Network

# Create a PyVis network
net = Network()

# Add nodes and edges
net.add_nodes([1, 2, 3, 4])
net.add_edges([(1, 2), (1, 3), (2, 3), (2, 4), (3, 4)])

# Generate HTML
net.show("mygraph.html") 
```

**Choosing the Right Package:**

*   **NetworkX:** Best for general graph analysis and visualization tasks.
*   **Matplotlib:** Provides more control over the appearance but requires more coding.
*   **PyVis:** Ideal for creating interactive and visually appealing web-based visualizations.

**Additional Tips:**

*   Start with simple examples and gradually increase complexity.
*   Explore the documentation and tutorials for each library to learn about its features and capabilities.
*   Consider using Jupyter Notebooks for interactive visualization and experimentation.

By using these libraries, undergraduate computer science students can effectively visualize graphs, explore graph algorithms, and gain a deeper understanding of graph theory concepts.
