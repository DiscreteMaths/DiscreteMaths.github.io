
\pagestyle{fancy}
\setmarginsrb{20mm}{0mm}{20mm}{25mm}{12mm}{11mm}{0mm}{11mm}
\lhead{MathsResource} \chead{Introduction to Calculus} \rhead{Tutorial Sheets} %\input{tcilatex}
\begin{document}
%=============================== %
\section*{Formula Sheet}
\subsection*{Notation:}
\begin{framed}
	\begin{multicols}{2}
		\begin{itemize}
			\item $\mathbb{R}$ - All real numbers positive and negative
			\item $\mathbb{R}^+$ - All positive real numbers including $0$
			\item $\mathbb{R}^-$ - All negative real numbers including $0$
			\\
			\phantomspace{here}
			\bigskip 
			\item $[a,b]$ - All real numbers $x$ such that $a \le x \le b$
			\item $(a,b)$ - All real numbers $x$ such that $a < x < b$
			\item $[a,\infty)$ - All real numbers $x$ such that $a \le x$
			\item $(a,\infty)$ - All real numbers $x$ such that $a < x$
		\end{itemize} 
	\end{multicols}
\end{framed}
	
\subsection*{Floor and Ceiling Functions}
	\begin{itemize}
		\item $\lceil x\rceil$ : Ceiling function
		\item $\lfloor x\rfloor$  : Floor Function
		\item $\{x\}$ : Fractional Part of a number
		($\{x\} = x- \lfloor x\rfloor$)
	\end{itemize}
	\subsection*{Logarithms}
	\begin{framed}
	If $a^b = c$ then $\mbox{log}_a c = b$.
	\end{framed}
	

	The following laws are very useful for working with logarithms.
	\begin{multicols}{2}
	\begin{enumerate}
		\item $\mbox{log}_b(X)$ + $\mbox{log}_b(Y)$ = $\mbox{log}_b(XY)$\bigskip
		\item $\mbox{log}_b(X)$ - $\mbox{log}_b(Y)$ = $\mbox{log}_b(X / Y)$ \bigskip
		\item $\mbox{log}_b(X^Y)$ = $Y \mbox{log}_b(X)$
		
		\item $\mbox{log}_b(X) = 1 $ when $b=X$
	\end{enumerate}
	\end{multicols}


\subsection*{Change of Base Formula}

\[ \log_A(B) = \frac{ \log_e(B) }{ \log_e(A) }  \]
	\subsection*{Sum and Difference of Two Cubes}
	\[ a^3 + b^3 = (a-b)(a^2 - ab + b^2)\]
	\[ a^3 - b^3 = (a-b)(a^2 + ab + b^2)\]
	
	%======================================== %
	
	\subsection*{Sequences and Series}
	




Arithmetic Series Summation:
\begin{multicols}{2}	
	\[ \sum_{i=1}^{n} i = \frac{n(n+1)}{2}\]
	

	\[ S_n = \frac{n}{2} \left(2a + (n-1) d \right)\]
\end{multicols}	
	Geometric Series Summation:
	\begin{multicols}{2}
	\[ S_n = a\left(\frac{1-r^n}{1-r}\right)\]
	
	\[ S_\infty = \frac{a}{1-r}\]
	\end{multicols}
	
	
	\subsection*{Ratio Test}
	
	For a series with general term $u_n$, if
	
	\[ \lim_{n \to \infty } \left| \frac{u_{n+1}}{u_n} \right| = r\]
	then the series converges (absolutely) if $r<1$
	
	
	%==========================================================================================%
	
\newpage
	\subsection*{Curve Sketching}
	\begin{description}
		\item[Horizontal Asymptote:] The horizontal asymptote is computed as
		\[ \lim_{x \to \infty } f(x) \]
	\end{description}

	%==========================================================================================%
	
			\subsection*{Maclaurin Series}
			\[f(x) = f(0) + f^{\prime}(0) + \frac{f^{\prime \prime}(0)}{2!} + \frac{f^{\prime \prime}(0)}{2!} + \frac{f^{\prime\prime \prime}(0)}{3!} + \ldots \]
	%==========================================================================================%
	
	\subsection*{Hyperbolic Functions }
	
	\[ \cosh(x)  =  \frac{e^{x} + e^{-x}}{2} \]
	
	\[ \sinh(x)  = \frac{e^{x} - e^{-x}}{2} \]
	
	%==========================================================================================%

	\subsection*{Rules of Differentiation}
	%% - http://media.wix.com/ugd/b064dd_62ab7d7000c34b7eb4a8e2108269cf27.pdf
	\begin{description}
		\item[Product Rule:]  with $y = uv$
		
		
		\[ \frac{dy}{dx} = u \frac{dv}{dx} +  v \frac{du}{dx} \]
		
		\item[Quotient Rule:] \[ y = \frac{u}{v}\]
		\[ \frac{dy}{dx}  = \frac{v \frac{du}{dx} - u \frac{dv}{dx} }{v^2} \]
		
		
		
		\item[Chain Rule:]
		
		% y = f(u) and u = u(x), that is y = f(u(x)) ⇒ dy
		% dx
		\[ \frac{dy}{dx} = \frac{dy}{du} \times \frac{du}{dx}  \]
	\end{description}

	\subsection*{Integration}
	
	Integration by parts: 
	
	\[ \int u dv = uv - \int v du \]  
	

	
	\subsection*{Dynamics}
	Where $s(t)$ denotes displacement at time $t$, $v(t)$ denotes the velocity at time $t$ and $a(t)$
	denotes the acceleration at time $t$, 
	\begin{multicols}{2}
	\[  \frac{ds(t)}{dt}  = v(t),\]
	\[  \frac{dv(t)}{dt}  = a(t).\]
	\end{multicols}
	\subsection*{Electrical Circuits}
	Where $q(t)$ denotes the charge at time $t$ and $i(t)$ denotes the current at time $t$,
	\[  \frac{dq(t)}{dt}  = i(t).\]
	


	
	
\subsection*{Curve Sketching}
\begin{description}
	\item[Horizontal Asymptote:] The horizontal asymptote is computed as
		\[ \lim_{x \to \infty } f(x) \]
\end{description}

	
	\subsection*{Notation}
	%% - http://media.wix.com/ugd/b064dd_62ab7d7000c34b7eb4a8e2108269cf27.pdf


%%%%%%%%%%%%%%%%%%%%%%%%%%%%%%%%%%%%%%%%%%%%%%%%%%%
\begin{framed}
	\begin{multicols}{2}
		\begin{itemize}
			\item $\mathbb{R}$ - All real numbers positive and negative
			\item $\mathbb{R}^+$ - All positive real numbers including $0$
			\item $\mathbb{R}^-$ - All negative real numbers including $0$
			\item $[a,b]$ - All real numbers $x$ such that $a \le x \le b$
			\item $(a,b)$ - All real numbers $x$ such that $a < x < b$
			\item $[a,\infty)$ - All real numbers $x$ such that $a \le x$
			\item $(a,\infty)$ - All real numbers $x$ such that $a < x$
		\end{itemize} 
	\end{multicols}
\end{framed}
\newpage

\newpage
	\section*{Formula Sheet}
	
\subsection*{Logarithms}
If $a^b = c$ \, then \, $\mbox{log}_a c = b$.



\subsection*{Change of Base Formula}

\[ \log_A(B) = \frac{ \log_e(B) }{ \log_e(A) }  \]
	\subsection*{Sum and Difference of Two Cubes}
	\[ a^3 + b^3 = (a-b)(a^2 - ab + b^2)\]
	\[ a^3 - b^3 = (a-b)(a^2 + ab + b^2)\]
	
	%======================================== %
	
	\subsection*{Sequences and Series}
Arithmetic Series Summation:
\begin{multicols}{2}	
	\[ \sum_{i=1}^{n} i = \frac{n(n+1)}{2}\]
	

	\[ S_n = \frac{n}{2} \left(2a + (n-1) d \right)\]
\end{multicols}	
	Geometric Series Summation:
	\begin{multicols}{2}
	\[ S_n = a\left(\frac{1-r^n}{1-r}\right)\]
	
	\[ S_\infty = \frac{a}{1-r}\]
	\end{multicols}
	
	
	\subsection*{Ratio Test}
	
	For a series with general term $u_n$, if
	
	\[ \lim_{n \to \infty } \left| \frac{u_{n+1}}{u_n} \right| = r\]
	then the series converges (absolutely) if $r<1$
	
	
	%==========================================================================================%
	
	
	\subsection*{Curve Sketching}
	\begin{description}
		\item[Horizontal Asymptote:] The horizontal asymptote is computed as
		\[ \lim_{x \to \infty } f(x) \]
	\end{description}
	%==========================================================================================%
	
	\subsection*{Maclaurin Series}
	\[f(x) = f(0) + f^{\prime}(0) + \frac{f^{\prime \prime}(0)}{2!} + \frac{f^{\prime\prime \prime}(0)}{3!} + \ldots \]
	%==========================================================================================%
	
	\subsection*{Hyperbolic Functions }
	
	\begin{multicols}{2}
	\[ \cosh(x)  =  \frac{e^{x} + e^{-x}}{2} \]
	
	\[ \sinh(x)  = \frac{e^{x} - e^{-x}}{2} \]
	\end{multicols}

	
	%==========================================================================================%
	

	\subsection*{Integration}
	
	Integration by parts: 
	
	\[ \int u dv = uv - \int v du \]  
	
\noindent Further formulae and special cases on pages 25 \& 26 of the log tables provided.
	
	\subsection*{Dynamics}
	Where $s(t)$ denotes displacement at time $t$, $v(t)$ denotes the velocity at time $t$ and $a(t)$
	denotes the acceleration at time $t$, 
	\begin{multicols}{2}
	\[  \frac{ds(t)}{dt}  = v(t),\]
	\[  \frac{dv(t)}{dt}  = a(t).\]
	\end{multicols}
	\subsection*{Electrical Circuits}
	Where $q(t)$ denotes the charge at time $t$ and $i(t)$ denotes the current at time $t$,
	\[  \frac{dq(t)}{dt}  = i(t).\]
	
	

\end{document}
