	%%- \frametitle{The Permutation Formula}
	The number of different permutations of r items from n unique items is written as $^n P_k$
	
	
	\[ ^n P_k = \frac{n!}{(n-k)!}\]

	%%- \frametitle{Permutations}
	\textbf{Example:}
	How many ways are there of arranging 3 different jobs, between 5 workers, where each worker can only do one job?
	
	
	\[ ^5 P_3 = \frac{5!}{(5-3)!}  = {5! \over 2!} = 60\]
	

	\subsection{Combinations}
	In mathematical terms, a combination is an subset of items from a larger set such that the order of the items does not matter.

\subsection{Permutations}

There are two types of permutation:
\begin{enumerate}
	\item Repetition is Allowed: such as the lock above. It could be "333".
	\item No Repetition: for example the first three people in a running race. You can't be first and second.
\end{enumerate}



\begin{framed}
\begin{itemize}
	\item Permutations where repetition is allowed: 
	\[ n! \]
	\item Permutations where repetition is not allowed
	\[ \frac{n!}{(n-k)!} \]
\end{itemize}
\end{framed}

	\noindent \textbf{Worked Example}
	A committee of 4 must be chosen from 3 females and 4 males.
	
	\begin{itemize}
		\item[$\bullet$] In how many ways can the committee be chosen.
		\item[$\bullet$] In how many can 2 males and 2 females be chosen.
		\item[$\bullet$]  Compute the probability of a committee of 2 males and 2 females are chosen.
		\item[$\bullet$] Compute the probability of at least two females.
	\end{itemize}
	\bigskip
}


%======================================================== %
%%- \frametitle{Using \texttt{R}}
%When implementing combination calculations in \texttt{R}, we use the \texttt{choose()} function.
%
%\begin{verbatim}
%> choose(5,0)
%[1] 1
%> choose(5,1)
%[1] 5
%> choose(5,2)
%[1] 10
%> choose(5,3)
%[1] 10
%> choose(5,4)
%[1] 5
%> choose(5,5)
%[1] 1
%\end{verbatim}



\subsection{Permutations}


\textbf{Part 2 : PERMUTE}\\
\begin{itemize}
	\item We re-express the answer from part 2 as follows:
	
	\[\frac{7!}{2!} =  \frac{5040}{2} = \boldsymbol{2520} \]
\end{itemize}

\textbf{Part 4 : LITTLE}\\
\begin{itemize}
	\item The word LITTLE has 6 letters, but there are two Ls and two Ts.
	\item From before, there are 6! ways to arrange 6 letters.
	\item Again, interchanging the two Ls and Ts does not result in a new permutation. 
	
	\[\frac{6!}{2!\times 2!} =  \frac{720}{4} = \boldsymbol{180} \]
\end{itemize}
%====================================================================================%

\begin{itemize}
	\item In how many permutations are there of counting a subset of k elements, when there are $n$ elements in total.
	
	\item The number of permutations of a set of n elements is denoted n! (pronounced n factorial.)
\end{itemize}

%------------------------------------------------------ %
	\subsection{Permutations}
	
	\begin{itemize}
		\item The notion of permutation relates to the act of permuting (rearranging) objects or values. 
		\item Informally, a permutation of a set of objects is an arrangement of those objects into a particular order. 
		
		\item For example, there are six permutations of the set $\{1,2,3\}$, namely (1,2,3), (1,3,2), (2,1,3), (2,3,1), (3,1,2), and (3,2,1). 
		\item As another example, an anagram of a word is a permutation of its letters. 
		
	\end{itemize}
	
	If the probability of C is $70 \%$ then the probability of $C^{\prime}$ is $30\%$		
	
	%================================================================ %

\subsection*{Permutations : Worked Example}

How many anagrams (permutations of the letters) are there of the following words

\begin{framed}
\begin{multicols}{2}
\begin{enumerate}
	\item ANSWER
	\item PERMUTE
	\item ANAGRAM
	\item LITTLE
\end{enumerate}
\end{multicols}
\end{framed}


\textbf{Part 1 : ANSWER}\\
Some possible outcomes:
\begin{center}
	ASNWRE,\;
	SANERW,\;
	REWSAN,\;...
\end{center}

Since ANSWER has 6 distinct letters, the number of permutations (anagrams) is

\[6! = 6\times 5 \times 4 \times 3 \times 2\times 1 = \boldsymbol{720} \]

%==================================================================================%
\textbf{Part 2 : PERMUTE}\\
\begin{itemize}
	\item[$\bullet$] The word PERMUTE has 7 letters, but only 6 different letters. 
	\item[$\bullet$] There are 7! ways to arrange 7 letters.
	\item[$\bullet$] However, interchanging the two Es does not result in a new permutation. There would be two identical anagrams.
\end{itemize}

\begin{center}
	P\textcolor{red}{E}RMUT\textcolor{blue}{E}, \; MUT\textcolor{red}{E}P\textcolor{blue}{E}R, \; P\textcolor{red}{E}T\textcolor{blue}{E}MUR,\; ..\\
	P\textcolor{blue}{E}RMUT\textcolor{red}{E}, \; MUT\textcolor{blue}{E}P\textcolor{red}{E}R, \; P\textcolor{blue}{E}T\textcolor{red}{E}MUR,\; ..
\end{center}


\begin{itemize}
	\item[$\bullet$]  The number of permutations (anagrams) is half of 7! .
	
	\[\frac{7!}{2} =  \frac{5040}{2} = \boldsymbol{2520} \]
\end{itemize}

%==================================================================================%	

\textbf{Part 3 : ANAGRAM}\\
\begin{itemize}
	\item The word ANAGRAM has 7 letters, but there are three As.
	\item From before, there are 7! ways to arrange 7 letters.
	\item How many new permutations are found by re-arranging the As?
\end{itemize}

\begin{multicols}{2}
	\begin{itemize}
		\item[(i)]	\textcolor{red}{A}N\textcolor{blue}{A}GR\textcolor{green}{A}M 
		\item[(ii)]		\textcolor{red}{A}N\textcolor{green}{A}GR\textcolor{blue}{A}M 
		\item[(iii)] 		\textcolor{blue}{A}N\textcolor{red}{A}GR\textcolor{green}{A}M  
		\item[(iv)]		\textcolor{green}{A}N\textcolor{red}{A}GR\textcolor{blue}{A}M 
		\item[(v)]	\textcolor{blue}{A}N\textcolor{green}{A}GR\textcolor{red}{A}M 
		\item[(vi)]		\textcolor{green}{A}N\textcolor{blue}{A}GR\textcolor{red}{A}M 
	\end{itemize}
\end{multicols}

\begin{itemize}
	\item We divide 7! by 3! to account for the identical anagrams.
	
	\[\frac{7!}{3!} =  \frac{5040}{6} = \boldsymbol{840} \]
\end{itemize}

%-------------------------------------%



\subsection{Permutation Formula}

A formula for the number of possible permutations of k objects from a set of n. This is usually written $^nP_k$ .


\noindent\textbf{Formula:}	
\[ ^nP_k = \frac{n!}{(n-k)!} =  n.(n-1).(n-2).\ldots(n-k+1) \]



\noindent \textbf{Example:}\\	
How many ways can 4 students from a group of 15 be lined up for a photograph?\\

%--------------- %
\noindent \textbf{Answer:	}\\
There are $^{15}P_4$ possible permutations of 4 students from a group of 15.
\[ ^{15}P_4 = \frac{15!}{11!} = 15\times 14\times 13\times 12 = 32760 \]
There are 32760 different lineups.







