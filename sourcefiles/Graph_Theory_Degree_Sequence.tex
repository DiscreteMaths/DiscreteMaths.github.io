**Short Questions on Degree Sequences in Graph Theory:**

**Why include them in Tutorial Sheets for Computer Science Undergraduates?**

* **Foundation Building:** Degree sequences are a fundamental concept in graph theory. These questions help students grasp the basic definition and properties of degrees in both undirected and directed graphs.
* **Conceptual Understanding:** 
    * Questions about the possibility of a given sequence being the degree sequence of a graph test students' understanding of the underlying constraints and relationships between vertex degrees. 
    * For example, the Handshaking Lemma (sum of degrees is even for undirected graphs) is implicitly tested.
* **Problem-Solving Skills:** 
    * These questions encourage students to think critically and apply the definitions to analyze given sequences. 
    * They develop analytical and problem-solving skills crucial for computer science.
* **Real-world Relevance:** Degree sequences have applications in network analysis, social network analysis, and other areas where graph models are used.
* **Assessment:** 
    * These questions are easy to grade and provide valuable feedback to both students and instructors. 
    * They help assess students' understanding of basic graph theory concepts.

**Example Questions:**

* **Q1:** Is it possible for an undirected graph to have a degree sequence of (3, 3, 2, 2, 1)? 
    * **Solution:** Yes. The sum of degrees is even, and it's possible to construct such a graph.
* **Q2:** Is it possible for an undirected graph to have a degree sequence of (4, 3, 2, 1)?
    * **Solution:** No. The sum of degrees is odd, which violates the Handshaking Lemma for undirected graphs.
* **Q3:** What are the possible in-degree and out-degree sequences for a directed graph with 4 vertices? 
    * **Solution:** There are various possibilities. For example, in-degrees: (2, 1, 1, 0), out-degrees: (1, 0, 1, 2) is a valid combination.

**Incorporating Solutions:**

* **Self-Assessment:** Providing solutions allows students to self-assess their understanding and identify areas where they need improvement.
* **Learning Tool:** Solutions can serve as a guide for students to understand the reasoning behind the answers and learn from their mistakes.
* **Instructor Feedback:** Instructors can use the solutions to quickly assess student understanding and identify areas where the class may need further instruction.

In conclusion, including short questions on degree sequences in tutorial sheets is beneficial for computer science undergraduates. These questions are effective for reinforcing fundamental concepts, developing problem-solving skills, and providing valuable feedback to both students and instructors.
