**Arithmetic Progressions and Series**

**What is an Arithmetic Progression?**

* An arithmetic progression (AP) is a sequence of numbers where the difference between consecutive terms is constant. 
* This constant difference is called the **common difference** (often denoted by 'd').

**Example:**

The sequence 3, 7, 11, 15, 19... is an arithmetic progression with a common difference of 4.

**General Term of an Arithmetic Progression:**

* The nth term of an arithmetic progression can be found using the formula:

  `a_n = a_1 + (n - 1)d`

  where:
    * `a_n` is the nth term
    * `a_1` is the first term
    * `n` is the position of the term
    * `d` is the common difference

**Arithmetic Series:**

* An arithmetic series is the sum of the terms in an arithmetic progression.

**Sum of an Arithmetic Series:**

* The sum of the first 'n' terms of an arithmetic series can be calculated using the formula:

  `S_n = (n/2)(2a_1 + (n - 1)d)`

  or

  `S_n = (n/2)(a_1 + a_n)`

  where:
    * `S_n` is the sum of the first 'n' terms
    * `a_1` is the first term
    * `a_n` is the nth term
    * `n` is the number of terms
    * `d` is the common difference

**Example:**

Find the sum of the first 10 terms of the arithmetic progression 2, 5, 8, 11...

* In this case, `a_1 = 2`, `d = 3`, and `n = 10`.
* Using the formula: `S_n = (n/2)(2a_1 + (n - 1)d)`
* `S_10 = (10/2)(2 * 2 + (10 - 1) * 3)`
* `S_10 = 5(4 + 27)`
* `S_10 = 5 * 31`
* `S_10 = 155`

Therefore, the sum of the first 10 terms of the given arithmetic progression is 155.

**Key Points:**

* Arithmetic progressions have a consistent pattern of adding the same value to each term.
* The formulas for the nth term and the sum of an arithmetic series are essential for solving problems related to these sequences.
* Arithmetic progressions have numerous applications in various fields, including finance, physics, and computer science.
