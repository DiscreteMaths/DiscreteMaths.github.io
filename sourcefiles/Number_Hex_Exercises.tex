**1. Number Representation**

*   **a) Decimal to Hexadecimal:**
    *   Divide 1234 by 16 repeatedly:
        *   1234 / 16 = 77 with a remainder of 2
        *   77 / 16 = 4 with a remainder of 13 (D in hexadecimal)
        *   4 / 16 = 0 with a remainder of 4
    *   Read the remainders from bottom to top: 4D2
    *   **Therefore, 1234 (decimal) = 0x4D2 (hexadecimal)**

*   **b) Hexadecimal to Decimal:**
    *   0xCAFE = (C * 16^3) + (A * 16^2) + (F * 16^1) + (E * 16^0)
    *   = (12 * 4096) + (10 * 256) + (15 * 16) + (14 * 1)
    *   = 49152 + 2560 + 240 + 14
    *   **Therefore, 0xCAFE (hexadecimal) = 51966 (decimal)**

*   **c) Hexadecimal Addition:**
    *   Align the numbers:
        ```
        0x2A
        + 0x5B
        -----
        ```
    *   Add the digits, handling carry-overs:
        *   A + B = 21 (15 in decimal) = F with a carry of 1
        *   2 + 5 + 1 (carry) = 8
    *   **Therefore, 0x2A + 0x5B = 0x85**

*   **d) Hexadecimal Subtraction:**
    *   Align the numbers:
        ```
        0x7F
        - 0x3A
        -----
        ```
    *   Subtract the digits, borrowing when necessary:
        *   F - A = 5 (borrow 1 from 7)
        *   6 (7 - 1 borrow) - 3 = 3
    *   **Therefore, 0x7F - 0x3A = 0x45**

**2. Bitwise Operations**

*   **a) Given A = 0xAB (10101011), B = 0xCD (11001101)**
    *   **AND (A & B):** 10001001 = 0x89
    *   **OR (A | B):** 11101111 = 0xEF
    *   **XOR (A ^ B):** 00100110 = 0x26
    *   **NOT (~A):** 01010100 = 0x54

*   **b) Using Bitwise Operations:**
    *   **Check if a bit is set:** AND the number with a mask where only the desired bit is set to 1. If the result is non-zero, the bit is set.
    *   **Set a bit:** OR the number with a mask where only the desired bit is set to 1.
    *   **Clear a bit:** AND the number with a mask where the desired bit is set to 0.
    *   **Toggle a bit:** XOR the number with a mask where only the desired bit is set to 1.

**3. Applications**

*   **a) Significance of Hexadecimal:**
    *   **Memory Addresses:** Compact representation for large memory addresses, easily convertible to binary.
    *   **Color Representation:** RGB and RGBA colors are often represented in hexadecimal for brevity and ease of use in web design and graphics.

*   **b) Hexadecimal in Networking:**
    *   **MAC Addresses:** Unique identifiers for network devices, often represented in hexadecimal.
    *   **IPv6 Addresses:** Uses hexadecimal notation for its 128-bit addresses.

*   **c) Advantages and Disadvantages of Hexadecimal:**
    *   **Advantages:** More compact than binary, easier to read and write than long binary strings.
    *   **Disadvantages:** Less intuitive for humans than decimal for everyday arithmetic.

**4. Challenging Problem:**

*   **a) Decimal to Hexadecimal Conversion Algorithm (Python):**

```python
def decimal_to_hexadecimal(decimal):
    hexadecimal = ""
    while decimal > 0:
        remainder = decimal % 16
        if remainder < 10:
            hexadecimal = str(remainder) + hexadecimal
        else:
            hexadecimal = chr(ord('A') + remainder - 10) + hexadecimal
        decimal = decimal // 16
    return hexadecimal

# Example usage:
decimal_number = 1234
hexadecimal_number = decimal_to_hexadecimal(decimal_number)
print(f"{decimal_number} (decimal) = 0x{hexadecimal_number} (hexadecimal)")
```

This algorithm iteratively divides the decimal number by 16, converts the remainders to their hexadecimal equivalents (0-9, A-F), and concatenates them to form the final hexadecimal string.
